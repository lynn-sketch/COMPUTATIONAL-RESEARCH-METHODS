\documentclass[12pt, a4paper]{article}
\usepackage{fullpage}
\usepackage{graphicx}
\usepackage{natbib}
\usepackage{amsmath}
\usepackage{hyperref}  

\begin{document}

\begin{center}

{\Large \textbf{UGANDA }}
\vspace{0.2cm}

{\Large \textbf{CHRISTIAN} \hspace{0.2cm} \includegraphics[width=0.1\textwidth]{ucu.jpg}
\hspace{0.2cm} \textbf{UNIVERSITY}}
\vspace{4cm}

{\LARGE \textbf{Research Paper}}
\vspace{2cm}
\end{center}

\begin{center}
\textbf{\textit{by}}
\vspace{1cm}
\end{center}

\begin{center}

\begin{tabular}{|l|l|l|}
\hline
\textbf{Name} & \textbf{RegNo} & \textbf{AccessNo} \\ \hline
Amoit Lynn & M23B38/007 & B20813 \\ \hline
Lakica Leticia & M23B23/023 & B20720 \\ \hline
Atwijukirwe Apophia & M23B23/051 & B23592 \\ \hline
\end{tabular}

\end{center}

\newpage

\title{Study on Artificial Intelligence: The State of the Art and Future Prospects}
\author{Caiming Zhang, Yang Lu}
\date{\today}
\maketitle

\begin{abstract}
In the world, the technological and industrial revolution is accelerating by the widespread application of new generation information and communication technologies, such as AI, IoT (the Internet of Things), and blockchain technology. Artificial intelligence has attracted much attention from government, industry, and academia. In this study, popular articles published in recent years that relate to artificial intelligence are selected and explored. This study aims to provide a review of artificial intelligence based on industry information integration. It presents an overview of the scope of artificial intelligence using background, drivers, technologies, and applications, as well as logical opinions regarding the development of artificial intelligence. This paper may play a role in AI-related research and should provide important insights for practitioners in the real world. The main contribution of this study is that it clarifies the state of the art of AI for future study.
\end{abstract}

\newpage  

\section{Introduction}
In 1956, at a conference at Dartmouth University, scholars formally proposed the term "artificial intelligence." That moment was the first step in a new topic of studying how machines simulate human intelligent activities. In early 2016, AlphaGo defeated the world chess champion. This event immediately aroused global interest in artificial intelligence (AI) \cite{zhang2021}. The development of artificial intelligence has brought huge economic benefits to mankind and has benefited all aspects of life, even as it has greatly promoted social development and brought social development into a new era \cite{ding2020}. 

AI is the general term for the science of artificial intelligence. It uses computers to simulate human intelligent behaviors and it trains computers to learn human behaviors such as learning, judgment, and decision-making \cite{feng2001}. AI is a knowledge project that takes knowledge as the object, acquires knowledge, analyzes and studies the expression methods of knowledge, and employs these approaches to achieve the effect of simulating human intellectual activities \cite{li2001}. AI plays an indispensable role in social development, and it has brought revolutionary results in improving labor efficiency, reducing labor costs, optimizing the structure of human resources, and creating new job demands \cite{hamet2017}.

\newpage  

\section{The Origin and Development of Artificial Intelligence}
Artificial intelligence is the study of how to make computers perform intelligent tasks that, in the past, could only be performed by humans \cite{zhang2021}. In recent years, AI has developed rapidly, and it has changed people's lifestyles \cite{ding2020}. The development of AI has become an important development strategy for countries around the world, enhancing national competitiveness and maintaining security \cite{duan2019}. Many countries have introduced preferential policies and have strengthened the deployment of key technologies.

\section{Big Data}
Big data is a prerequisite for AI, and it is a core factor that promotes AI to improve recognition rate and accuracy. With the development and the wide application of the Internet of Things, the amount of data generated has increased exponentially, with a great increased annual growth rate. In addition to increasing the number, the dimensionality of the data has also been expanded \cite{duan2019b}. These large amounts of high-dimensional data make the data more comprehensive and more sufficient to support AI applications.

\section{The Expert System}
An expert system is a knowledge system based on the existing knowledge of human experts. The expert system was the earliest field AI research. It is widely used in medical diagnosis, in geological surveying, and in the petrochemical industry. Expert systems usually refer to various knowledge systems \cite{feng2001}. This is an intelligent computer program, based on knowledge, that uses professional knowledge provided by human experts to simulate the thinking process of human experts and uses knowledge and logic to make decisions.

\section{Application Scenarios of AI}
Based on the relatively mature development of technical conditions such as data, algorithms, and computing capabilities, AI has begun to truly solve problems and to effectively create economic benefits \cite{finogeev2019}. From an application perspective, industries with a good data foundation (such as finance, healthcare, automotive, and retail) have relatively mature AI application scenarios \cite{hamet2017}.

\section{Three Viewpoints of AI}
Symbolism, connectionism, and behaviorism represent the three main viewpoints of concepts in the field of AI research. They are the most important theoretical theorems in the development of AI disciplines, and they are the theoretical foundation for the development of AI disciplines.

\section{Conclusion}
This study presents a systematic overview of AI that focuses on prospects and development, core techniques, applicable scenarios, and challenges. This paper conducts a state-of-the-art review of the extant and upcoming research on AI. Artificial intelligence is an interdisciplinary subject that involves information, logic, cognition, thinking, systems, and biology. It has been used for knowledge processing, pattern recognition, machine learning, and natural language processing. Applications have been successfully implemented in a variety of industries, enhancing productivity and decision-making.

\section*{References}
\bibliographystyle{plain}
\bibliography{mybib}  

\end{document}

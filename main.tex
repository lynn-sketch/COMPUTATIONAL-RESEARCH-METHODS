\documentclass[12pt, a4paper]{article}
\usepackage{fullpage}
\usepackage{graphicx}
\usepackage{natbib}
\usepackage{amsmath}
\usepackage{hyperref} % For clickable links

\begin{document}

\begin{center}

{\Large \textbf{UGANDA }}
\vspace{0.2cm}

{\Large \textbf{CHRISTIAN} \hspace{0.2cm} \includegraphics[width=0.1\textwidth]{ucu.jpg}
\hspace{0.2cm} \textbf{UNIVERSITY}}
\vspace{4cm}

{\large \textbf{Bridging the Information Gap Between Farmers and Consumers in Agriculture}}
\vspace{2cm}
\end{center}

\begin{center}
\textbf{\textit{by}}
\vspace{1cm}
\end{center}

\begin{center}

\begin{tabular}{|l|l|l|}
\hline
\textbf{Name}&\textbf{RegNo}&\textbf{AccessNo}\\ \hline
Amoit Lynn&M23B38/007&B20813\\ \hline
Lakica Leticia&M23B23/023&B20720\\ \hline
Atwijukirwe Apophia&M23B23/051&B23592\\ \hline
\end{tabular}
\vfill
\today

\end{center}

\newpage

\section*{Topics of Interest}

We sat down and discussed our interests as a group which included skin care, agriculture, and health. We then narrowed it down and decided to focus on agriculture.

\section*{Selected Research Topic}

Agriculture

\section*{Background Statement}

Agriculture is the predominant economic activity in rural areas, which harbors about 76 percent of Uganda's population. Farmers in rural areas face the challenge of efficiently matching consumer supply and demand, especially for perishable goods. This is often due to limited access to real-time market data regarding consumer demand for specific produce items, creating a gap that leads to significant inefficiencies in the food supply chain and negative consequences. Farmers find it difficult to make informed decisions regarding crops to grow, resulting in the possibility of oversupply or undersupply, spoilage, and missed market opportunities. Due to their perishable nature, many agricultural products are affected as delays in accessing market information can potentially lead to substantial losses. On the other hand, consumers face limited access to diverse produce and price volatility due to a lack of transparency and information, which undermines the sustainability and resilience of the agricultural sector, impacting both food security and the livelihood of farmers. Existing market information systems often fail to reach farmers in rural or underserved areas; they lack user-friendly interfaces or don't provide data tailored specifically to a farmer’s needs. Poor connections to markets impact farmers’ incomes directly and keep them in a cycle of low investment and productivity \citep{WorldBank2020}. Bridging this information gap is crucial for empowering farmers, reducing food waste, and ensuring a stable and accessible food supply.

\begin{center}
    \includegraphics[width=0.8\textwidth]{ProblemTree.png}
\end{center}

\newpage

\section*{Problem Statement}

A lack of transparent and readily available information about agricultural products hinders both farmers and consumers. Farmers struggle to make informed decisions about what crops to grow based on market demand and consumer preferences, keeping them in a cycle of low investment and productivity. Consumers, on the other hand, often lack information about the origin, quality, and production methods of the food they purchase, leading to mistrust and potentially limiting their ability to support sustainable and ethical farming practices.

\begin{center}
    \includegraphics[width=0.8\textwidth]{SolutionTree.png}
\end{center}

\section*{Existing Solutions}

\begin{enumerate}
    \item Mobile-based market apps
    \textit{Gaps:} Lack of localized data
    \item Exploring blockchain technology
    \textit{Gaps:} Cost and complexity
    \item Improving internet access in rural areas
    \textit{Gaps:} Neglection of digital literacy
\end{enumerate}

\section*{Proposed Solutions}

\begin{enumerate}
    \item Developing user-friendly market information platforms
    \begin{itemize}
        \item \textit{SDG 12: Responsible Consumption and Production:} By reducing food waste and optimizing agricultural production, these platforms contribute to more sustainable consumption and production patterns in the food system.
        \item \textit{Vision 2040: Economic Growth:} Most national development plans emphasize the importance of agricultural development for overall economic growth. By improving market access and productivity for farmers, these platforms contribute to the growth of the agricultural sector and its contribution to the national economy.
    \end{itemize}
    \item Connecting farmers directly with consumers through farmer market partnerships/groups
    \begin{itemize}
        \item \textit{SDG 2: Zero Hunger:} Direct connections reduce post-harvest losses by providing immediate market access for perishable goods. This increases food availability and reduces waste, contributing to food security. It also allows farmers to diversify and grow what consumers demand, improving nutrition.
        \item \textit{Vision 2040: Agricultural Transformation:} Vision 2040 prioritizes agricultural transformation. Direct market access empowers farmers to modernize their operations, increase productivity, and become more commercially oriented. This shift from subsistence to commercial farming is crucial for agricultural transformation.
    \end{itemize}
\end{enumerate}

\citep{Obaideen2022}
\citep{Guggisberg2022}

\bibliographystyle{plainnat} 
\bibliography{mybib} 

\end{document}